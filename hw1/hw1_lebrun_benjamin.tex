%%%%%%%%%%%%%%%%%%%%%%%%%%%%%%%%%%%%%%%%%
% Memo
% LaTeX Template
% Version 1.0 (30/12/13)
%
% This template has been downloaded from:
% http://www.LaTeXTemplates.com
%
% Original author:
% Rob Oakes (http://www.oak-tree.us) with modifications by:
% Vel (vel@latextemplates.com)
%
% License:
% CC BY-NC-SA 3.0 (http://creativecommons.org/licenses/by-nc-sa/3.0/)
%
%%%%%%%%%%%%%%%%%%%%%%%%%%%%%%%%%%%%%%%%%

\documentclass[letterpaper,11pt]{texMemo} % Set the paper size (letterpaper, a4paper, etc) and font size (10pt, 11pt or 12pt)

\usepackage{parskip} % Adds spacing between paragraphs
\usepackage[colorlinks]{hyperref}
\usepackage{graphicx}
\usepackage{float}
\usepackage{listings}
\hypersetup{citecolor=DeepPink4}
\hypersetup{linkcolor=red}
\hypersetup{urlcolor=blue}
\usepackage{cleveref}
\setlength{\parindent}{15pt} % Indent paragraphs

\lstset{
  language=C,                % choose the language of the code
  numbers=left,                   % where to put the line-numbers
  stepnumber=1,                   % the step between two line-numbers.        
  numbersep=5pt,                  % how far the line-numbers are from the code
  backgroundcolor=\color{white},  % choose the background color. You must add \usepackage{color}
  showspaces=false,               % show spaces adding particular underscores
  showstringspaces=false,         % underline spaces within strings
  showtabs=false,                 % show tabs within strings adding particular underscores
  tabsize=2,                      % sets default tabsize to 2 spaces
  captionpos=b,                   % sets the caption-position to bottom
  breaklines=true,                % sets automatic line breaking
  breakatwhitespace=true,         % sets if automatic breaks should only happen at whitespace
  title=\lstname,                 % show the filename of files included with \lstinputlisting;
}

%----------------------------------------------------------------------------------------
%	MEMO INFORMATION
%----------------------------------------------------------------------------------------

%----------------------------------------------------------------------------------------
%	MEMO INFORMATION
%----------------------------------------------------------------------------------------

\memoto{Dr. Larry Pyeatt} % Recipient(s)

\memofrom{Benjamin Lebrun} % Sender(s)

\memosubject{Homework 1} % Memo subject

\memodate{\today} % Date, set to \today for automatically printing todays date

%\logo{\includegraphics[width=0.1\textwidth]{logo.png}} % Institution logo at the top right of the memo, comment out this line for no logo

%----------------------------------------------------------------------------------------

\begin{document}


\maketitle % Print the memo header information

%----------------------------------------------------------------------------------------
%	MEMO CONTENT
%----------------------------------------------------------------------------------------

\section*{Problem 1.2}
Perform the base conversions to fill out the table
\subsection*{Solution}

\begin{table}[ht]
    \begin{tabular}{lllll}
    Base 10        & Base 2             & Base 16       & Base 21 &  \\
    $23_{10}$      & $10111_2$          & $17_{16}$     & $12_{21}$      &  \\
    $19_{10}$      & $010011_2$         & $13_{16}$     & $J_{21}$       &  \\
    $2747_{10}$    & $0101010111011_2$  & $ABB_{16}$    & $64H_{21}$     &  \\
    $1253_{10}$    & $010011100101_2$   & $4E5_{16}$    & $2HE_{21}$     & 
    \end{tabular}
\end{table}

This was done by following the process in the book via exponents to convert to base 10, then back using
long division.

Example of converting $2HE_{21}$ to base 10.

\[
    2*21^2+17*21^1+14*21^0 = 1253
\]

And an example of the conversion of $ABB_{16}$ to base 21 (after finding the decimal equilvalent 2747).

\[
    \frac{2747}{21} = 130 R 17
\]

\[
    \frac{130}{21} = 6 R 4
\]

\[
    \frac{6}{21} = 0 R 6
\]

Taking the remainders, we reconstruct the base 21 number as $64H$

\newpage
\section*{Problem 1.4}
What is \textbackslash minus ! given that \textbackslash and ! are ASCII charadcers? Give your answer in binary.
\subsection*{Solution}
Find the characters integer values in ASCII, subtract, then convert to binary.
\textbackslash = 92
! = 33

\[
    92 - 33 = 59
\]

$\frac{59}{2} = 29 R 1$

$\frac{29}{2} = 14 R 1$

$\frac{14}{2} = 7 R 0$

$\frac{7}{2} = 3 R 1$

$\frac{3}{2} = 1 R 1$

$\frac{1}{2} = 0 R 1$

This makes the answer $111011_2$

Code was also generated to test this output.

\lstinputlisting{test.c}

\newpage
\section*{Problem 1.6}
Suppose that the string "This is a nice day" is stored beginning at address 
$4B3269AC_{16}$. What are the contents of the byte at address $4B3269B1_{16}$

\subsection*{Solution}
In hex, we can simply subtract the last two digits of the address with each other, since in the subtraction,
the remaining digits cancel. Therefore, when we convert $AC$ and $B1$ back into binary we find 
177 and 172, giving us a difference of 5. Assuming every character is the size of one address,
at the address $4B3269B1_{16}$ we would find the character 'i' in ASCII, which in hex is 69.

\section*{Problem 1.8}
Given the following binary string:

01001001 01110011 01101110 00100111 01110100 00100000 01000001 01110011 01110011 01100101 01101101 01100010
01101100 01111001 00100000 01000110 01110101 01101110 00111111 0000000

\subsection*{Hexadecimal string}
This can be done quickly by splitting each 8 bit string and splitting it in two, then solving for each side of
the Hexadecimal equilvalent for that string.

49 73 6E 27 74 20 41 73 73 65 6D 62 6C 79 20 46 75 6E 3F 00

\subsection*{First four bytes to a stirng of base ten numbers}
73 115 110 39

\subsection*{Convert the first (little-endian) halfword to base ten}
Little endian begins at the least significant byte. The last byte in our string is the string of zeros in base 2.

0

\subsection*{Convert the first (big-endian) halfword to base ten}
Big endian begins at the most significant byte. In our case is the very first byte in the string.

73

\subsection*{Convert the binary string to ASCII characters}
Isn't Assembly Fun?\textbackslash0

\newpage
\section*{Problem 1.11}
UTF-8 is often referred to as Unicode. Why is that not correct?

\subsection*{Solution}
Unicode only has a 16-bit encoding, UTF-8 however is variable length encoding.

\section*{Problem 1.13}
What are the differences between a CPU register adn a memory location.

\subsection*{Solution}
A CPU register is a piece of memory that actually sits on the chip, generally awaiting to be operated
on with a command. The other generally sits on RAM with an address to be accessed and loaded onto a register.

\end{document}