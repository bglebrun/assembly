%%%%%%%%%%%%%%%%%%%%%%%%%%%%%%%%%%%%%%%%%
% Memo
% LaTeX Template
% Version 1.0 (30/12/13)
%
% This template has been downloaded from:
% http://www.LaTeXTemplates.com
%
% Original author:
% Rob Oakes (http://www.oak-tree.us) with modifications by:
% Vel (vel@latextemplates.com)
%
% License:
% CC BY-NC-SA 3.0 (http://creativecommons.org/licenses/by-nc-sa/3.0/)
%
%%%%%%%%%%%%%%%%%%%%%%%%%%%%%%%%%%%%%%%%%

\documentclass[letterpaper,11pt]{texMemo} % Set the paper size (letterpaper, a4paper, etc) and font size (10pt, 11pt or 12pt)

\usepackage{parskip} % Adds spacing between paragraphs
\usepackage[colorlinks]{hyperref}
\usepackage{graphicx}
\usepackage{float}
\usepackage{listings}
\usepackage{enumitem}
\hypersetup{citecolor=DeepPink4}
\hypersetup{linkcolor=red}
\hypersetup{urlcolor=blue}
\usepackage{cleveref}
\setlength{\parindent}{15pt} % Indent paragraphs

\lstset{
  language=C,                % choose the language of the code
  numbers=left,                   % where to put the line-numbers
  stepnumber=1,                   % the step between two line-numbers.        
  numbersep=5pt,                  % how far the line-numbers are from the code
  backgroundcolor=\color{white},  % choose the background color. You must add \usepackage{color}
  showspaces=false,               % show spaces adding particular underscores
  showstringspaces=false,         % underline spaces within strings
  showtabs=false,                 % show tabs within strings adding particular underscores
  tabsize=2,                      % sets default tabsize to 2 spaces
  captionpos=b,                   % sets the caption-position to bottom
  breaklines=true,                % sets automatic line breaking
  breakatwhitespace=true,         % sets if automatic breaks should only happen at whitespace
  title=\lstname,                 % show the filename of files included with \lstinputlisting;
}

%----------------------------------------------------------------------------------------
%	MEMO INFORMATION
%----------------------------------------------------------------------------------------

%----------------------------------------------------------------------------------------
%	MEMO INFORMATION
%----------------------------------------------------------------------------------------

\memoto{Dr. Larry Pyeatt} % Recipient(s)

\memofrom{Benjamin Lebrun} % Sender(s)

\memosubject{Homework 3} % Memo subject

\memodate{\today} % Date, set to \today for automatically printing todays date

%\logo{\includegraphics[width=0.1\textwidth]{logo.png}} % Institution logo at the top right of the memo, comment out this line for no logo

%----------------------------------------------------------------------------------------

\begin{document}


\maketitle % Print the memo header information

%----------------------------------------------------------------------------------------
%	MEMO CONTENT
%----------------------------------------------------------------------------------------

\section*{Problem 3.1}
Which registers hold the stack pointer, return address, and program counter?
\subsection*{Solution}
The stack pointer sits on register 13, the return address sits on register 14 in the link 
register, and the program counter on register 15.


\section*{Problem 3.4}
The stm and ldm instructions include an optional ‘!’ after the address register. What does
it do?
\subsection*{Solution}
If the '!' Is included, the address of the operation is stored back into the register.

\section*{Problem 3.5}
The following C statement declares an array of four integers, and initializes their values
to 7, 3, 21, and 10, in that order.
\begin{lstlisting}
    int nums[]={7,3,21,10};
\end{lstlisting}
\begin{enumerate}[label=\Alph*]
    \item Write the equivalent in GNU ARM assembly
    \item Write the ARM assembly instructions to load all four numbers into registers r3, r5, r6, and r9, respectively, using:
    i. a single ldm instruction, and
    ii. four ldr instructions.
\end{enumerate}
\subsection*{Solution}
\begin{lstlisting}
nums: .word 7, 3, 21, 10 @ Array of values
\end{lstlisting}

\begin{lstlisting}
    ldm =nums,{r3-r6,r9}
    ldr r3,[=nums]
    ldr r4,[=nums,1]
    ldr r5,[=nums,2]
    ldr r6,[=nums,3]
    ldr r9,[=nums,4]
\end{lstlisting}

\section*{Problem 3.6}
What is the difference between a memory location and a CPU register?
\subsection*{Solution}
A memory location generally sits on the stack which is usually in RAM, we move the memory location, or pointer to memory, to the register
to be operated on by the CPU, then usually stored back into memory.

\newpage
\section*{Problem 3.8}
Use ldm and stm to write a short sequence of ARM assembly language to copy 16 words of data from a source address to a
destination address. Assume that the source address is already loaded in r0 and the destination address is already loaded in r1.
You may use registers r2 through r5 to hold values as needed. Your code is allowed to modify r0 and/or r1.
\subsection*{Solution}
\lstinputlisting{3.8.S}


\section*{Problem 3.10}
Assume that x is an array of integers, and i and j are integers. Convert the following C statements into ARM assembly language.
\begin{enumerate}[label=\Alph*]
    \item x[i] = j;
    \item x[j] = x[i];
    \item x[i] = x[j*2];
\end{enumerate}
\subsection*{Solution}
\lstinputlisting{3.10.S}


\newpage
\section*{Problem 3.11}
What is the difference between the b instruction and the bl instruction? What is each used for?
\subsection*{Solution}
The bl instruction saves the address into the link register to return to, otherwise functionally
both instructions causes the code to branch to a different section of code.
\end{document}