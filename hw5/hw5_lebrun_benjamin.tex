%%%%%%%%%%%%%%%%%%%%%%%%%%%%%%%%%%%%%%%%%
% Memo
% LaTeX Template
% Version 1.0 (30/12/13)
%
% This template has been downloaded from:
% http://www.LaTeXTemplates.com
%
% Original author:
% Rob Oakes (http://www.oak-tree.us) with modifications by:
% Vel (vel@latextemplates.com)
%
% License:
% CC BY-NC-SA 3.0 (http://creativecommons.org/licenses/by-nc-sa/3.0/)
%
%%%%%%%%%%%%%%%%%%%%%%%%%%%%%%%%%%%%%%%%%

\documentclass[letterpaper,11pt]{texMemo} % Set the paper size (letterpaper, a4paper, etc) and font size (10pt, 11pt or 12pt)

\usepackage{parskip} % Adds spacing between paragraphs
\usepackage[colorlinks]{hyperref}
\usepackage{graphicx}
\usepackage{float}
\usepackage{listings}
\usepackage{enumitem}
\hypersetup{citecolor=DeepPink4}
\hypersetup{linkcolor=red}
\hypersetup{urlcolor=blue}
\usepackage{cleveref}
\setlength{\parindent}{15pt} % Indent paragraphs

\lstset{
  language=C,                % choose the language of the code
  numbers=left,                   % where to put the line-numbers
  stepnumber=1,                   % the step between two line-numbers.        
  numbersep=5pt,                  % how far the line-numbers are from the code
  backgroundcolor=\color{white},  % choose the background color. You must add \usepackage{color}
  showspaces=false,               % show spaces adding particular underscores
  showstringspaces=false,         % underline spaces within strings
  showtabs=false,                 % show tabs within strings adding particular underscores
  tabsize=2,                      % sets default tabsize to 2 spaces
  captionpos=b,                   % sets the caption-position to bottom
  breaklines=true,                % sets automatic line breaking
  breakatwhitespace=true,         % sets if automatic breaks should only happen at whitespace
  title=\lstname,                 % show the filename of files included with \lstinputlisting;
}

%----------------------------------------------------------------------------------------
%	MEMO INFORMATION
%----------------------------------------------------------------------------------------

%----------------------------------------------------------------------------------------
%	MEMO INFORMATION
%----------------------------------------------------------------------------------------

\memoto{Dr. Larry Pyeatt} % Recipient(s)

\memofrom{Benjamin Lebrun} % Sender(s)

\memosubject{Homework 5} % Memo subject

\memodate{\today} % Date, set to \today for automatically printing todays date

%----------------------------------------------------------------------------------------

\begin{document}


\maketitle % Print the memo header information

%----------------------------------------------------------------------------------------
%	MEMO CONTENT
%----------------------------------------------------------------------------------------

\section*{Problem 5.1}
What does it mean for a register to be volatile? Which ARM registers are considered
volatile according to the ARM function calling convention?
\subsection*{Solution}
A register that is volatile means that it's available to be freely used, therefore
never attempt to store important data to a volatile register and always either
allocate data to the stack, from a global variable, or use one of the non-volatile
registers r4-r11 but saving these registers to the stack or to a variable before
changing them within a function.

\section*{Problem 5.2}
Fully explain the differences between static variables and automatic variables.
\subsection*{Solution}
Automatic variables only stick around as long as the current block of code is active
and within scope. This variable is destroyed when we go out of scope of the code block.
A static variable however will attepmt to remain alive for the duration of the program's
lifetime.

\newpage
\section*{Problem 5.5}
You are writing a function and you decided to use registers r4 and r5 within the 
function. Your function will not call any other functions; it is self-contained. 
Modify the following skeleton structure to ensure that r4 and r5 can be used within 
the function and are restored to comply with the ARM standards, but without unnecessary
memory accesses.
\begin{lstlisting}
    myfunc: stmfd   sp!,{lr}

    @ function statements

            ldmfd   sp!,{lr}
\end{lstlisting}
\subsection*{Solution}

\begin{lstlisting}
    myfunc: stmfd   sp!,{lr}
            push    {r4-r5}

    @ function statements

            pop     {r4-r5}
            ldmfd   sp!,{lr}
\end{lstlisting}


\section*{Problem 5.6}
Convert the following C program to ARM assembly, using a post-test loop:
\begin{lstlisting}
int main() {
    for(i=0;i<10;i++) printf("Hi!\n");
    return 0;
}
\end{lstlisting}
\subsection*{Solution}
\lstinputlisting{5.6.S}

\newpage
\section*{Problem 5.8}
Write a complete subroutine in ARM assembly that is equivalent to the following C subroutine.
\begin{lstlisting}
/* This function copies 'count' bytes from 'src' to 'dest'. */
void bytecopy(char des[], char src[], int count) {
    count = count - 1;
    while(count>=0) {
        dest[count] = src[count];
        count = count - 1;
    }
}
\end{lstlisting}
\subsection*{Solution}
\lstinputlisting{5.8.S}

\newpage
\section*{Problem 5.10}
Write an ARM assembly function to calculate the average of an array of integers, given a
pointer to the array and the number of items in the array. Your assembly function must 
implement the following C function prototype:
\begin{lstlisting}
int average(int *array, int number_of_items);
\end{lstlisting}
Assume that the processor does not support the div instruction, but there is a function
available to divide two integers. You do not have to write this function, but you may need
to call it. Its C prototype is:
\begin{lstlisting}
int divide(int numberator, int denominator);
\end{lstlisting}
\subsection*{Solution}
\lstinputlisting{5.10.S}

\section*{Problem 5.11}
Write a complete function in ARM assembly that is equivalent to the following C function.
Note that a and b must be allocated on the stack, and their addresses must be passed to
scanf so that it can place their values into memory.
\begin{lstlisting}
int read_and_add() {
    int a, b, sum;
    scanf("%d",&a);
    scanf("%d",&b);
    sum = a + b;
    return sum;
}
\end{lstlisting}
\subsection*{Solution}
\lstinputlisting{5.11.S}

\section*{Problem 5.13}
For large x, the factorial function is slow. However, a lookup table can be added to the
function to improve average performance. This technique is commonly known as memoization
or tabling, but is sometimes called dynamic programming. The following C implementation
of the factorial function uses memoization. Modify your ARM assembly program from the
previous problem to include memoization.
\begin{lstlisting}
#define TABSIZE 50
/* The factorial function calculates x! */
int factorial(int x) {
    /* declare table and initialize to all zero */
    static in table[TABSIZE] = {0};

    /* handle base case */
    if(x<2) return 1;

    /* if x! is not in the table and x is  small enough,
       then compute x! and put it in the table */
       if((x < TABSIZE) && (table[x] == 0)) table[x] = x * factorial(x-1);

    /* if x is small enough then return the value from the table */
    if(x < TABSIZE) return table[x];

    /* if x is too large to be in the table, use a recursive call */
    return x * factorial(x - 1);
}
\end{lstlisting}
\subsection*{Solution}
\lstinputlisting{5.13.S}

\end{document}