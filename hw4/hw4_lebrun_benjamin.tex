%%%%%%%%%%%%%%%%%%%%%%%%%%%%%%%%%%%%%%%%%
% Memo
% LaTeX Template
% Version 1.0 (30/12/13)
%
% This template has been downloaded from:
% http://www.LaTeXTemplates.com
%
% Original author:
% Rob Oakes (http://www.oak-tree.us) with modifications by:
% Vel (vel@latextemplates.com)
%
% License:
% CC BY-NC-SA 3.0 (http://creativecommons.org/licenses/by-nc-sa/3.0/)
%
%%%%%%%%%%%%%%%%%%%%%%%%%%%%%%%%%%%%%%%%%

\documentclass[letterpaper,11pt]{texMemo} % Set the paper size (letterpaper, a4paper, etc) and font size (10pt, 11pt or 12pt)

\usepackage{parskip} % Adds spacing between paragraphs
\usepackage[colorlinks]{hyperref}
\usepackage{graphicx}
\usepackage{float}
\usepackage{listings}
\usepackage{enumitem}
\hypersetup{citecolor=DeepPink4}
\hypersetup{linkcolor=red}
\hypersetup{urlcolor=blue}
\usepackage{cleveref}
\setlength{\parindent}{15pt} % Indent paragraphs

\lstset{
  language=C,                % choose the language of the code
  numbers=left,                   % where to put the line-numbers
  stepnumber=1,                   % the step between two line-numbers.        
  numbersep=5pt,                  % how far the line-numbers are from the code
  backgroundcolor=\color{white},  % choose the background color. You must add \usepackage{color}
  showspaces=false,               % show spaces adding particular underscores
  showstringspaces=false,         % underline spaces within strings
  showtabs=false,                 % show tabs within strings adding particular underscores
  tabsize=2,                      % sets default tabsize to 2 spaces
  captionpos=b,                   % sets the caption-position to bottom
  breaklines=true,                % sets automatic line breaking
  breakatwhitespace=true,         % sets if automatic breaks should only happen at whitespace
  title=\lstname,                 % show the filename of files included with \lstinputlisting;
}

%----------------------------------------------------------------------------------------
%	MEMO INFORMATION
%----------------------------------------------------------------------------------------

%----------------------------------------------------------------------------------------
%	MEMO INFORMATION
%----------------------------------------------------------------------------------------

\memoto{Dr. Larry Pyeatt} % Recipient(s)

\memofrom{Benjamin Lebrun} % Sender(s)

\memosubject{Homework 4} % Memo subject

\memodate{\today} % Date, set to \today for automatically printing todays date

%\logo{\includegraphics[width=0.1\textwidth]{logo.png}} % Institution logo at the top right of the memo, comment out this line for no logo

%----------------------------------------------------------------------------------------

\begin{document}


\maketitle % Print the memo header information

%----------------------------------------------------------------------------------------
%	MEMO CONTENT
%----------------------------------------------------------------------------------------

\section*{Problem 4.1}
If r0 initially contains 1, what will it contain after the third instruction in the sequence below?

\begin{lstlisting}
    add r0,r0,#1
    mov r1,r0
    add r0,r1,r0    lsl #1
\end{lstlisting}
\subsection*{Solution}
r0 will contain 6 after the third instruction execution.

\section*{Problem 4.2}
What will r0 and r1 contain after each of the following instructions? Give your answers in base 10.
\begin{lstlisting}
    mov r0,#1
    mov r1,#0x20    @ load 32
    orr r1,r1,r0
    lsl r1,#0x2     @ shift 2, 132
    orr r1,r1,r0
    eor r0,r0,r1    @ xor
    lsr r1,r0,#3
\end{lstlisting}
\subsection*{Solution}
r0 is 132 and r1 is 16 after these instructions execute

\newpage

\section*{Problem 4.3}
What is the difference between lsr and asr?
\subsection*{Solution}
asr is generally for signed numbers, and takes into account if the number being shifted is positive or
negative, but fill the bits with the appropriate compliment. lsr will shift, but will always fill the 
bits with zero instead of the compliment or signed value.

\section*{Problem 4.5}
Given the following variable definitions:
\begin{lstlisting}
    num1:   .word   x
    num2:   .word   y
\end{lstlisting}
where you do not know the values of x and y, write a short sequence of ARM assembly instructions to load
the two numbers, compare them, and move the largest number into register r0.
\subsection*{Solution}
\begin{lstlisting}
    ldr     r0,=num1    @ load r0, x
    ldr     r1,=num2    @ load r1, y
    cmp     r0,r1       @ compare r0 <= r1
    movgt   r0,r1       @ r0 = r1; else largest is already in r0
\end{lstlisting}

\section*{Problem 4.6}
Assuming that a is stored in register r0 and b is stored in register r1, show the ARM assembly code that
is equivalent to the following C code.
\begin{lstlisting}
    if (a & 1 )
        a = -a;
    else
        b = b+7;
\end{lstlisting}
\subsection*{Solution}
\begin{lstlisting}
    tst     r0,#1       @ test 0 bit of a
    negne   r0          @ negate if CPSR flag set
    addeq   r1,r1,#7    @ else add 7 to b
\end{lstlisting}

\section*{Problem 4.7}
Without using the mul instruction, give the instructions to multiply r3 by the following constants, leaving
the result in r0. You may also use r1 and r2 to hold temporary results, and you do not need to preserve the
original contents of r3.
\begin{enumerate}[label=\Alph*]
    \item 10
    \item 100
    \item 575
    \item 123
\end{enumerate}
\subsection*{Solution}
Using shift and add algorithm

A.
\begin{lstlisting}
    mov r2, r3
    lsl r3, r3, #3
    lsl r2, r2, #5
    add r0, r2, r3
\end{lstlisting}
B.
\begin{lstlisting}
    mov r2, r3
    mov r1, r2
    lsl r3, r3, #6
    lsl r2, r2, #5
    lsl r1, r1, #2
    add r0, r2, r3
    add r0, r0, r1
\end{lstlisting}
C.
\begin{lstlisting}
    mov r2, r3
    mov r1, r2
    lsl r3, r3, #9
    lsl r2, r2, #6
    lsl r1, r1, #1
    sub r2, r2, r1
    add r0, r2, r3
\end{lstlisting}
D.
\begin{lstlisting}
    mov r2, r3
    lsl r3, r3, #7
    lsl r2, r2, #1
    sub r0, r3, r2
\end{lstlisting}

\section*{Problem 4.8}
Assume that r0 holds the least significant 32 bits of a 64-bit integer a, and r1 holds the most significant 32
bits of a. Likewise, r2 holds the least significant 32 bits of a 64-bit integer b, and r3 holds the most
significant 32 bits of b. Show the shortest instruction sequences necessary to:
\begin{enumerate}[label=\Alph*]
    \item compare a to b, setting the CPSR flags,
    \item shift a left by one bit, storing the result in b,
    \item add b to a, and
    \item subtract b from a.
\end{enumerate}
\subsection*{Solution}
A.
\begin{lstlisting}
    cmp     r1, r3
    cmpeq   r0, r2
\end{lstlisting}
B.
\begin{lstlisting}
    lsls    r2, r0, #1
    lsl     r3, r1, #1
    addcs   r3, r3, #1
\end{lstlisting}
C.
\begin{lstlisting}
    adds r2, r2, r0
    adc  r3, r3, r1
\end{lstlisting}
D.
\begin{lstlisting}
    subs r2, r2, r0
    sbc  r3, r3, r1
\end{lstlisting}

\end{document}