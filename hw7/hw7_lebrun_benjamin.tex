%%%%%%%%%%%%%%%%%%%%%%%%%%%%%%%%%%%%%%%%%
% Memo
% LaTeX Template
% Version 1.0 (30/12/13)
%
% This template has been downloaded from:
% http://www.LaTeXTemplates.com
%
% Original author:
% Rob Oakes (http://www.oak-tree.us) with modifications by:
% Vel (vel@latextemplates.com)
%
% License:
% CC BY-NC-SA 3.0 (http://creativecommons.org/licenses/by-nc-sa/3.0/)
%
%%%%%%%%%%%%%%%%%%%%%%%%%%%%%%%%%%%%%%%%%

\documentclass[letterpaper,11pt]{texMemo} % Set the paper size (letterpaper, a4paper, etc) and font size (10pt, 11pt or 12pt)

\usepackage{parskip} % Adds spacing between paragraphs
\usepackage[colorlinks]{hyperref}
\usepackage{graphicx}
\usepackage{float}
\usepackage{listings}
\usepackage{enumitem}
\hypersetup{citecolor=DeepPink4}
\hypersetup{linkcolor=red}
\hypersetup{urlcolor=blue}
\usepackage{cleveref}
\setlength{\parindent}{15pt} % Indent paragraphs

\lstset{
  language=C,                % choose the language of the code
  numbers=left,                   % where to put the line-numbers
  stepnumber=1,                   % the step between two line-numbers.        
  numbersep=5pt,                  % how far the line-numbers are from the code
  backgroundcolor=\color{white},  % choose the background color. You must add \usepackage{color}
  showspaces=false,               % show spaces adding particular underscores
  showstringspaces=false,         % underline spaces within strings
  showtabs=false,                 % show tabs within strings adding particular underscores
  tabsize=2,                      % sets default tabsize to 2 spaces
  captionpos=b,                   % sets the caption-position to bottom
  breaklines=true,                % sets automatic line breaking
  breakatwhitespace=true,         % sets if automatic breaks should only happen at whitespace
  title=\lstname,                 % show the filename of files included with \lstinputlisting;
}

%----------------------------------------------------------------------------------------
%	MEMO INFORMATION
%----------------------------------------------------------------------------------------

%----------------------------------------------------------------------------------------
%	MEMO INFORMATION
%----------------------------------------------------------------------------------------

\memoto{Dr. Larry Pyeatt} % Recipient(s)

\memofrom{Benjamin Lebrun} % Sender(s)

\memosubject{Homework 5} % Memo subject

\memodate{\today} % Date, set to \today for automatically printing todays date

%----------------------------------------------------------------------------------------

\begin{document}


\maketitle % Print the memo header information

%----------------------------------------------------------------------------------------
%	MEMO CONTENT
%----------------------------------------------------------------------------------------

\section*{Problem 7.1}
Multiply -90 by 105 using signed 8-bit binary multiplication to form a signed 16-bit
result. Show all of your work.
\subsection*{Solution}
See attached worksheet.

\section*{Problem 7.2}
Multiply 166 by 105 using unsigned 8-bit binary multiplication to form an unsigned 16-bit
result. Show all of your work.
\subsection*{Solution}
See attached worksheet.

\section*{Problem 7.3}
Write a section of ARM assembly code to multiply the value in r1 by 13 using only shift
and add operations.
\subsection*{Solution}
\begin{lstlisting}
    add r0,r1,r1,lsl #2
    add r0,r0,r1,lsl #3
\end{lstlisting}

\newpage

\section*{Problem 7.4}
The following code will multiply the value in r0 by a constant C. What is C?
\begin{lstlisting}
    add     r1,r0,r0,lsl #1
    add     r0,r1,r0,lsl #2
\end{lstlisting}
\subsection*{Solution}
C = 7

\section*{Problem 7.6}
Show how to divide 78 by 6 using binary long division.
\subsection*{Solution}
See attached worksheet.

\section*{Problem 7.7}
Demonstrate the division algorithm using a sequence of tables as shown in Section 7.3.2
to divide 155 by 11.
\subsection*{Solution}
See attached worksheet.

\end{document}