%%%%%%%%%%%%%%%%%%%%%%%%%%%%%%%%%%%%%%%%%
% Memo
% LaTeX Template
% Version 1.0 (30/12/13)
%
% This template has been downloaded from:
% http://www.LaTeXTemplates.com
%
% Original author:
% Rob Oakes (http://www.oak-tree.us) with modifications by:
% Vel (vel@latextemplates.com)
%
% License:
% CC BY-NC-SA 3.0 (http://creativecommons.org/licenses/by-nc-sa/3.0/)
%
%%%%%%%%%%%%%%%%%%%%%%%%%%%%%%%%%%%%%%%%%

\documentclass[letterpaper,11pt]{texMemo} % Set the paper size (letterpaper, a4paper, etc) and font size (10pt, 11pt or 12pt)

\usepackage{parskip} % Adds spacing between paragraphs
\usepackage[colorlinks]{hyperref}
\usepackage{graphicx}
\usepackage{float}
\usepackage{listings}
\usepackage{enumitem}
\hypersetup{citecolor=DeepPink4}
\hypersetup{linkcolor=red}
\hypersetup{urlcolor=blue}
\usepackage{cleveref}
\setlength{\parindent}{15pt} % Indent paragraphs

\lstset{
  language=C,                % choose the language of the code
  numbers=left,                   % where to put the line-numbers
  stepnumber=1,                   % the step between two line-numbers.        
  numbersep=5pt,                  % how far the line-numbers are from the code
  backgroundcolor=\color{white},  % choose the background color. You must add \usepackage{color}
  showspaces=false,               % show spaces adding particular underscores
  showstringspaces=false,         % underline spaces within strings
  showtabs=false,                 % show tabs within strings adding particular underscores
  tabsize=2,                      % sets default tabsize to 2 spaces
  captionpos=b,                   % sets the caption-position to bottom
  breaklines=true,                % sets automatic line breaking
  breakatwhitespace=true,         % sets if automatic breaks should only happen at whitespace
  title=\lstname,                 % show the filename of files included with \lstinputlisting;
}

%----------------------------------------------------------------------------------------
%	MEMO INFORMATION
%----------------------------------------------------------------------------------------

%----------------------------------------------------------------------------------------
%	MEMO INFORMATION
%----------------------------------------------------------------------------------------

\memoto{Dr. Larry Pyeatt} % Recipient(s)

\memofrom{Benjamin Lebrun} % Sender(s)

\memosubject{Homework 2} % Memo subject

\memodate{\today} % Date, set to \today for automatically printing todays date

%\logo{\includegraphics[width=0.1\textwidth]{logo.png}} % Institution logo at the top right of the memo, comment out this line for no logo

%----------------------------------------------------------------------------------------

\begin{document}


\maketitle % Print the memo header information

%----------------------------------------------------------------------------------------
%	MEMO CONTENT
%----------------------------------------------------------------------------------------

\section*{Problem 2.1}
What is the difference between
\begin{enumerate}[label=\Alph*]
    \item the .data section and .bss section?
    \item the .ascii and .asciz directives?
    \item the .word and the .long directives?
\end{enumerate}
\subsection*{Solution}
\begin{enumerate}[label=\Alph*]
    \item .bss is generally filled with zeroes for initializing certain variables, mainly anything that needs to be initialized to zero;
        .data is generally used for global variables and constants with labels.
    \item .asciz is like the .ascii directive which expects zero or more string literals. .asciz although follows
        each string with an ASCII null character.
    \item for the ARM processor, these two directives act exactly the same. However, in another assembly language these two
        may simply represent different sized data types.
\end{enumerate}

\section*{Problem 2.2}
What is the purpose of the .align assembler directive? What does ".align 2" do in GNU ARM assembly?
\subsection*{Solution}
For the ARM CPU, if a word is not properly aligned the processor will take more time to load if it is not
properly aligned. Proper alignment will allow the CPU to load one byte, two bytes(half word) and four bytes(word)
quickly.

.align 2 will fill an area on the processor with two extra zero padding bytes, if put correctly after a 
label or variable of data will provide a performance boost. This is generally good practice to put an align
directive after declaring a character or half word of data.

\section*{Problem 2.5}
What is the total memory, in bytes, allocated for the following variables?
\begin{lstlisting}
    var1: .word 23
    var2: .long 0xC
    expr: .ascii ">>"
\end{lstlisting}
\subsection*{Solution}
\begin{enumerate}
    \item 4 byte
    \item 4 bytes
    \item 3 bytes
\end{enumerate}

\section*{Problem 2.7}
Write assembly code to declare variables equivalent to the following C code:
\begin{lstlisting}
    /* these variables are declared outside of any function */
    static int foo[3];  /* visible anywhere in the current file */
    static char bar[4]; /* visible anywhere in the current file */
    char barfoo;        /* visible anywhere in the program */
    int foobar;         /* visible anywhere in the program */
\end{lstlisting}
\subsection*{Solution}
\begin{lstlisting}
    foo:    .word   0x000
    bar:    .ascii  "\0\0\0\0"
    barfoo: .ascii  "\0"
    foobar: .word   0
\end{lstlisting}

\newpage
\section*{Problem 2.9}
Insert the minimum number of .align directives necessary in the following code
so that all word variables are aligned on word boundaries and all halfword variables
are aligned on halfword boundaries, while minimizing the amount of wasted space.
\begin{lstlisting}
        .data
        .align  2
    a:  .byte   0
    b:  .word   32
    c:  .byte   3
    d:  .hword  45
    e:  .hword  0
    f:  .byte   0
    g:  .word   128
\end{lstlisting}
\subsection*{Solution}
\begin{lstlisting}
    .data
    .align  2
a:  .byte   0
b:  .word   32
c:  .byte   3
    .align  1
d:  .hword  45
    .align  4
e:  .hword  0
f:  .byte   0
g:  .word   128
\end{lstlisting}

\section*{Problem 2.12}
Why are .ascii and .asciz directives usually followed by an .align directive, but .word directives are not?
\subsection*{Solution}
For performance on the ARM processor, we need the address aligned to a four byte boundary, or on a two byte boundary
for half words. If our .ascii or .asciz data does not end on this boundary, it's important to use the .align directive
after declaring this data to realign the program to the four or two byte boundary.

\newpage
\section*{Problem 2.13}
Using the “Hello World” program shown in Listing 2.1 as a template, write a program that will print your name.
\subsection*{Solution}
\lstinputlisting{print_name.S}
\end{document}