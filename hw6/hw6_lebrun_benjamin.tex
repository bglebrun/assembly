%%%%%%%%%%%%%%%%%%%%%%%%%%%%%%%%%%%%%%%%%
% Memo
% LaTeX Template
% Version 1.0 (30/12/13)
%
% This template has been downloaded from:
% http://www.LaTeXTemplates.com
%
% Original author:
% Rob Oakes (http://www.oak-tree.us) with modifications by:
% Vel (vel@latextemplates.com)
%
% License:
% CC BY-NC-SA 3.0 (http://creativecommons.org/licenses/by-nc-sa/3.0/)
%
%%%%%%%%%%%%%%%%%%%%%%%%%%%%%%%%%%%%%%%%%

\documentclass[letterpaper,11pt]{texMemo} % Set the paper size (letterpaper, a4paper, etc) and font size (10pt, 11pt or 12pt)

\usepackage{parskip} % Adds spacing between paragraphs
\usepackage[colorlinks]{hyperref}
\usepackage{graphicx}
\usepackage{float}
\usepackage{listings}
\usepackage{enumitem}
\hypersetup{citecolor=DeepPink4}
\hypersetup{linkcolor=red}
\hypersetup{urlcolor=blue}
\usepackage{cleveref}
\setlength{\parindent}{15pt} % Indent paragraphs

\lstset{
  language=C,                % choose the language of the code
  numbers=left,                   % where to put the line-numbers
  stepnumber=1,                   % the step between two line-numbers.        
  numbersep=5pt,                  % how far the line-numbers are from the code
  backgroundcolor=\color{white},  % choose the background color. You must add \usepackage{color}
  showspaces=false,               % show spaces adding particular underscores
  showstringspaces=false,         % underline spaces within strings
  showtabs=false,                 % show tabs within strings adding particular underscores
  tabsize=2,                      % sets default tabsize to 2 spaces
  captionpos=b,                   % sets the caption-position to bottom
  breaklines=true,                % sets automatic line breaking
  breakatwhitespace=true,         % sets if automatic breaks should only happen at whitespace
  title=\lstname,                 % show the filename of files included with \lstinputlisting;
}

%----------------------------------------------------------------------------------------
%	MEMO INFORMATION
%----------------------------------------------------------------------------------------

%----------------------------------------------------------------------------------------
%	MEMO INFORMATION
%----------------------------------------------------------------------------------------

\memoto{Dr. Larry Pyeatt} % Recipient(s)

\memofrom{Benjamin Lebrun} % Sender(s)

\memosubject{Homework 6} % Memo subject

\memodate{2/24/2020} % Date, set to \today for automatically printing todays date

%\logo{\includegraphics[width=0.1\textwidth]{logo.png}} % Institution logo at the top right of the memo, comment out this line for no logo

%----------------------------------------------------------------------------------------

\begin{document}


\maketitle % Print the memo header information

%----------------------------------------------------------------------------------------
%	MEMO CONTENT
%----------------------------------------------------------------------------------------
\section*{Problem 6.1}
What are the advantages of designing software using abstract data types?
\subsection*{Solution}
Designing software with hidden data types and allowing client code to interface with the data
instead is important in both modular programming, allowing us to split the work in teams, 
groups, induviduals, with the agreed interface in place, allowing the data behind the interface 
change and be modified or optimized without affecting other sections of code.

\section*{Problem 6.3}
High-level languages provide mechanisms for information hiding, but assembly does not.
Why should the assembly programmer not simply bypass all information hiding and access the
internal data structures of any ADT directly?
\subsection*{Solution}
Because future updates to the packages may move or modify how the internal ADT stores or organizes data.
If we're using assembly features such as pointers or attempting to directly access parts of the
ADTs memory, in a future update of the software, that accessible data may not actually exist. 
Breaking the assembly programmer's code.

\section*{Problem 6.5}
Given the following definitions for a stack ADT:
\lstinputlisting{./stack.h}
\lstinputlisting{./stack.c}
Write the InitStack() function in ARM assembly language
\subsection*{Solution}
\lstinputlisting{6.5.S}

\section*{Problem 6.6}
Referring to the previous question, write the Push() function in ARM assembly language.
\subsection*{Solution}
\lstinputlisting{6.6.S}

\section*{Problem 6.7}
Referring to the previous two questions, write the Pop() function in ARM assembly language.
\subsection*{Solution}
\lstinputlisting{6.7.S}

\section*{Problem 6.8}
Referring to the previous three questions, write the Top() function in ARM assembly language.
\subsection*{Solution}
\lstinputlisting{6.8.S}

\section*{Problem 6.9}
Referring to the previous four questions, write the PrintStack() function in ARM assembly language.
\subsection*{Solution}
\lstinputlisting{6.9.S}

\section*{Problem 6.11}
The “Software Engineering Code of Ethics And Professional Practice” states that a responsible software
engineer should “Approve software only if they have well-founded belief that it is safe, meets
specifications, passes appropriate tests…” (sub-principle 1.03) and “Ensure adequate testing, debugging,
and review of software…on which they work.” (sub-principle 3.10). Unfortunately, defects did make their
way into the system.
The software engineering code of ethics also states that a responsible software engineer should “Treat
all forms of software maintenance with the same professionalism as new development.”

\begin{enumerate}[label=(\alph*)]
    \item Explain how the Software Engineering Code of Ethics And Professional Practice were violated by the Therac 25 developers.
    \item How should the engineers and managers at AECL have responded when problems were reported?
    \item What other ethical and non-ethical considerations may have contributed to the deaths and injuries?
\end{enumerate}
\subsection*{Solution}
\begin{enumerate}[label=(\alph*)]
  \item The Therac 25 developers did not properly test their code on hardware, and could not reliably verify it was safe and meets specifications.
  Instead, leaving the code on machine without proper testing or debugging which allowed an entry error which showed settings which were
  incorrect to the specific patient.
  \item Engineers should have considered the software as part of the defects when investingating the machine. Instead,
  inputs to the software from hardware were considered issues instead of the software iteself.
  \item Failure to perform code review shows that little regard was ever given to the code writing process,
  especially when 
\end{enumerate}

\end{document}